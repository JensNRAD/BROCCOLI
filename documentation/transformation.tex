\section{Introduction}

BROCCOLI provides a separate function for transformation of volumes. In its simplest form, a transformation can with the bash wrapper be performed as

\begin{verbatim}
./TransformVolume volume_to_transform.nii reference_volume.nii ... 
displacement_field_x.nii displacement_field_y.nii displacement_field_z.nii
\end{verbatim}

where reference\_volume.nii is the reference volume that was used for the registration, and displacement\_field\_x.nii displacement\_field\_y.nii displacement\_field\_z.nii are generated by RegisterTwoVolumes.

\section{OpenCL options}

The following OpenCL options are available

\begin{itemize}

\item -platform
\newline \newline The OpenCL platform to use (default 0).

\item -device
\newline \newline The OpenCL device to use (default 0).

\end{itemize}

\newpage

\section{Transformation options}

The following transformation options are available

\begin{itemize}

\item -interpolation
\newline \newline The interpolation to use, 0 = nearest, 1 = trilinear (default 1). Nearest interpolation can for example be useful if you want to transform a binary mask.

\item -zcut
\newline \newline Number of mm to cut from the bottom of the input volume, can be negative (default 0). Should be the same as for the call to RegisterTwoVolumes.

\end{itemize}

\section{Outputs}

By default, the function saves the result as volume\_to\_transform\_warped.nii. 

\section{Output options}

The following output options are available

\begin{itemize}

\item -output 
\newline \newline Set output filename (default volume\_to\_transform\_warped.nii). 

\end{itemize}

\section{Additional options}

The following additional options are available

\begin{itemize}

\item -quiet 
\newline \newline Don't print anything to the terminal (default false). 

\end{itemize}


